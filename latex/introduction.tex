% $Id: introduction.tex 87303 2016-02-08 13:44:29Z lafferty $

\section{Introduction}
\label{sec:introduction}

CP violation enters the Standard Model (SM) via the single irreducible phase in the Cabbibo-Kobayashi-Maskawa (CKM) matrix. The unitarity requirement placed on this matrix allows us to construct the Unitray Triangle (UT) in the complex plane from the complex equation:
\begin{equation}
\Vud\Vubs+\Vcd\Vcbs+\Vtd\Vtbs=0
\end{equation}
This is of particular interest as all angles and sides are of measurable size and represent the weak phases and decay amplitudes respectively, of inter- and intra-generational quark transitions. The least well known angle, $\phi_3=arg(-\Vud\Vubs/\Vcd\Vcbs)$, is accessible
via tree-level processes, therefore, as well as being a fundamental parameter of the SM, a precise measurement of this angle at tree level can be compared to higher order processes in order to probe physics beyond the SM (BSM) that may enter decays via loops.

The \lhcb experiment has previously probed $\phi_3$ by studying the interference between $b\rightarrow c\bar{u}s$ and, the colour and CKM suppressed, $b\rightarrow u\bar{c}s$ transitions in $B^{\pm}\rightarrow DK$ decays, where the $D$ is an admixture of $D^{0}$ and $\bar{D}^{0}$ mesons decaying to a common final state. These could be two-body, non-self-conjugate multibody, or self-conjugate multibody final states. Similar studies have also been performed on neutral \Bz and \Bs mesons. This
analysis makes use of self-conjugate multibody final states, $D\rightarrow K^0_sh^+h^-$, in decays of the form
$B^{\pm}\rightarrow D^{*}K^{\pm}$, where the charmed meson is produced in an excited $D^*$ state and decays strongly via $D^{*0}\rightarrow D\pi^0 / \gamma$. Sensitivity to $\phi_3$ is then obtained by studying the Dalitz distribution of $D\rightarrow K^0_sh^+h^-$ events comparatively for $B^+$ and $B^-$ mesons. In order to achieve this, the strong phase variation of the $D$ meson decay over the bins of the Dalitz plot must be known.

% D*0 or D0 meson ??? If D*, the strong phase addition from the neutral is determined by its CP eigenvalue, which is +/- 1, respectively, for pi0/gamma strong D*0 decays

It is possible to determine this distribution using an amplitude model, derived from flavour-tagged $D\rightarrow K^0_sh^+h^-$ decays. This has been successfully applied in previous analyses by the \babar, \belle and \lhcb collaborations. The alternative approach is a model-independent method, as has been demonstrated by \belle and \lhcb, making use of direct  measurements of the strong phase
variation measured at CLEO-c in decays of $\psi(3770)$ mesons to quantum-correlated $D^0\bar{D}^0$ pairs. This data-driven approach is attractive as it avoids making the assumptions required by the model-dependant case.

This paper presents a model-independant study of $B^{\pm}\rightarrow D^*K^{\pm}$ decays, with $D^{*}\rightarrow D\pi^0$, $D^{*}\rightarrow D\gamma$ and $D\rightarrow K^0_s\pi^+\pi^-$, $D\rightarrow K^0_sK^+K^-$. The data used corresponds to a total integrated luminosity of $3.295$ fb$^{-1}$ of $pp$ collision acquired with the \lhcb detector, at center-of-mass energies of $\sqrt{s}=7,8,13$ \tev over the years 2011, 2012 and 2015, respectively.

~\cite{Alves:2008zz} ~\cite{REVTeX}

