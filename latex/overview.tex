\section{Analysis Overview}
We are following the discussion of Gershon \etal on $\Bm \to \Dstar h^-$.
The neutral \Dstar meson produced in the decay of the $\Bm$ is a superposition of particle, anti-particle states.
  The amplitudes of $\Bm \to \Dstarz h^-$ and $\Bm \to \Dstarzb h^-$ are
  \begin{eqnarray}
  A(\Bm \to \Dstarz h^-) = A_c e^{i\delta_c}&,&  A(\Bp \to \Dstarzb h^+) = A_c e^{i\delta_c} \\
A(\Bm \to \Dstarzb h^-) = A_u e^{i (\delta_u -\gamma)}&,&  A(\Bp \to \Dstarz h^+) = A_u e^{i(\delta_u + \gamma)} 
  \end{eqnarray}
 where $\delta_c$ and $\delta_u$ are strong phases, $\gamma$ is the weak phase.  
%The $\Bm \to \Dstar h^-$ amplitude can then be written as
 %\begin{eqnarray}
 %A(\Bm \to \Dstar h^-) = A_c e^{i\delta_c } [ \Dstarz + r_B e^{i (\delta_B -\gamma)} \Dstarzb ]  \\
 %A(\Bp \to \Dstar h^+) = A_c e^{i\delta_c } [\Dstarzb + r_B e^{i (\delta_B +\gamma)} \Dstarz].
 %\end{eqnarray} 
% where we could define
 % \begin{equation}
  %  \Dstar =\Dstarz+r_Be^{i(\delta_B-\gamma)}\Dstarzb.
    %\label{DstTilde}
  %\end{equation}
   We define CP eigenstates of the neutral \Dstar as $CP(\Dstarz) = \Dstarzb$, $CP(\Dstarzb) = \Dstarz$, then
\begin{eqnarray}
\Dstar_+ = \frac{\Dstarz + \Dstarzb}{\sqrt{2}}&,& \Dstar_- = \frac{\Dstarz - \Dstarzb}{\sqrt{2}} \\
\Dstarz = \frac{\Dstar_+ + \Dstar_-}{\sqrt{2}}&,& \Dstarzb = \frac{\Dstar_+ - \Dstar_-}{\sqrt{2}} 
\end{eqnarray}
%and we have 
 %\begin{equation}
  %  \Dstar = \frac{\Dstar_+ + \Dstar_-}{\sqrt{2}} + r_B e^{i (\delta_B -\gamma)} \frac{\Dstar_+ - \Dstar_-}{\sqrt{2}}.
   % \label{DstTildebar}
  %\end{equation}  
  
  According to Gershon \etal, the CP relationship between \Dstar and \D are $\Dstar_{\pm} \to D_{\mp} \gamma$ and $\Dstar_{\pm} \to D_{\pm} \pi^0$.
  Thus we have 
   \begin{eqnarray}
 A(\Bm \to (D\pi^0) h^-) &=& A_c e^{i\delta_c } A_{D^0 \pi^0} [ \Dz + r_B e^{i (\delta_B -\gamma)} \Dzb ]  \\
 A(\Bm \to  (D\gamma) h^-) &=& A_c e^{i\delta_c } A_{D^0 \gamma}[\Dz + r_B e^{i (\delta_B +\pi -\gamma)} \Dzb].
 \end{eqnarray} 

The \D decay amplitudes is defined as $A_D(p) = A_D(m_-^2,m_+^2)e^{i\delta_D(m_-^2, m_+^2)}$.
Assuming that CPV in \D decays is negligible and ignoring the effects of \D mixing, then $A_D(p) = A_{\bar{D}}(\bar{p})$,
we have 
 \begin{eqnarray}
 A(\Bm, \pi^0, h^-)  &=& A_c e^{i\delta_c } A_{D^0 \pi^0}[ A_D(p) + r_B e^{i (\delta_B -\gamma)} A_D(\bar{p}) ]  \\ 
A(\Bm, \gamma, h^-)  &=& A_c e^{i\delta_c } A_{D^0 \gamma} [A_D(p) + r_B e^{i (\delta_B +\pi -\gamma)} A_D(\bar{p})],
 \end{eqnarray} 
where we label the amplitude of $A(\Bm \to (K_S^0 h^+ h^-)_D h^0 h^-)$ in short as $A(\Bm, h^0, h^-)$.
The number of events in each Dalitz bin are then
 \begin{eqnarray}
 N_i(\Bm, \pi^0, h^-)  &=&  h_{\pi^0}\int  [ |A_D(p)|^2 + |r_B A_D(\bar{p})|^2  + 2r_B Re(A_D(p)A_D^*(\bar{p})e^{-i(\delta_B - \gamma)}) ] dp \\ 
N_i(\Bm, \gamma, h^-)  &=&h_{\gamma}\int  [|A_D(p)|^2 + |r_B A_D(\bar{p})|^2 - 2r_B Re(A_D(p)A_D^*(\bar{p})e^{-i(\delta_B - \gamma)}) ] dp
 \end{eqnarray} 
 where $h_{\pi^0}/h_{\gamma} = \BR(D^{*0} \to D^0 \pi^0)/\BR(D^{*0} \to D^0 \gamma)$.
The last term can be expressed as
\begin{equation}
Re[A_D(p)A_D^*(\bar{p})e^{-i(\delta_B - \gamma)}] = |A_D(p)A_D(\bar{p})| [\cos(\delta_D)\cos(\delta_B -\gamma) + \sin(\delta_D)\sin(\delta_B -\gamma) ]
\end{equation}
with
  \begin{equation}
      \Delta\delta_D \equiv \delta_D(m_+^2,m_-^2) - \delta_D(m_-^2,m_+^2)
    \end{equation}

The Dalitz plot is constructed in 2n bins labelled -n to +n (excluding 0), symmetric about the line $m_-^2\ =  m_+^2$, where N$>0$ corresponds to $m_-^2 > m_+^2$.
One then define 
\begin{eqnarray}
c_i &=& \frac{\int_i |A_D(p)A_D(\bar{p})|\cos(\delta_D)dp}{\sqrt{\int_i A^2_D(p)dp \int_i A^2_{\bar{D}}(p)dp}}\\
s_i  &=&  \frac{\int_i |A_D(p)A_D(\bar{p})|\sin(\delta_D)dp}{\sqrt{\int_i A^2_D(p)dp \int_i A^2_{\bar{D}}(p)dp}}\\
K_i &=& \int _i |A_D(p)|^2 dp, K_{\bar{i}} = \int _i |A_{\bar{D}}(p)|^2 dp
\end{eqnarray}
These variables are related by $c_i = c_{\bar{i}} = c_{-i}$, $s_i = -s_{\bar{i}} = -s_{-i}$ and $K_{-i} = K_{\bar{i}}$.
To assure proper normalization, we define 
\begin{equation}
F_i = \frac{K_i}{\sum_i K_i}
\end{equation}

We then have 
\begin{eqnarray}
 N_i(\Bm, \pi^0, h^-)  &=&  h_{\pi^0}( F_i + r^2_BF_{\bar{i}}  + 2\sqrt{F_iF_{\bar{i}}} (x_-c_i + y_-s_i ) )  \\
N_i(\Bm, \gamma, h^-)  &=& h_{\gamma}( F_i + r^2_BF_{\bar{i}}  - 2\sqrt{F_iF_{\bar{i}}} (x_-c_i + y_-s_i ) )
 \end{eqnarray} 
with $x_{\pm} = r_B \cos(\delta_B \pm \gamma)$ and $y_{\pm} = r_B \sin(\delta_B \pm \gamma)$.
$h_{\pi^0}$ and $h_{\gamma}$ is redefind to absorb $\sum_i K_i$ inside and $h_{\pi^0}/h_{\gamma} = \BR(D^{*0} \to D^0 \pi^0)/\BR(D^{*0} \to D^0 \gamma)$.

Taking into account efficiency affects, the above definition become
\begin{eqnarray}
c'_i &=& \frac{\int_i \epsilon(p) |A_D(p)A_D(\bar{p})|\cos(\delta_D)dp}{\sqrt{\int_i \epsilon(p) A^2_D(p)dp \int_i  \epsilon(p) A^2_{\bar{D}}(p)dp}}\\
s'_i  &=&  \frac{\int_i  \epsilon(p) |A_D(p)A_D(\bar{p})|\sin(\delta_D)dp}{\sqrt{\int_i \epsilon(p)  A^2_D(p)dp \int_i \epsilon(p) A^2_{\bar{D}}(p)dp}}\\
K'_i &=& \int _i \epsilon(p) |A_D(p)|^2 dp
\end{eqnarray}
In this case, the normalization requirement becomes
\begin{equation}
F'_i = \frac{K'_i}{\sum_i K'_i}
\end{equation}
\begin{eqnarray}
 N_i(\Bm, \pi^0, h^-)  &=&  h'_{\pi^0}( F'_i(\pi^0) + r^2_BF'_{\bar{i}}(\pi^0)  + 2\sqrt{F'_i(\pi^0)F'_{\bar{i}}(\pi^0)} (x_-c'_i(\pi^0) + y_-s'_i(\pi^0) ) )  \\
N_i(\Bm, \gamma, h^-)  &=& h'_{\gamma}( F'_i(\gamma) + r^2_BF'_{\bar{i}}(\gamma)  - 2\sqrt{F'_i(\gamma)F'_{\bar{i}}(\gamma)} (x_-c'_i(\gamma) + y_-s'_i(\gamma) ) )
 \end{eqnarray} 
and we have 
\begin{equation}
\frac{h'_{\pi^0}}{h'_{\gamma}} = \frac{\BR(D^{*0} \to D^0 \pi^0) \epsilon_{\textrm{ave}}(\pi^0)}{\BR(D^{*0} \to D^0 \gamma) \epsilon_{\textrm{ave}}(\gamma)}
\end{equation}
where the average efficiency for the decay is defined as
\begin{equation}
\epsilon_{\textrm{ave}} = \frac{\int \epsilon(p) |A_D(p)|^2 dp}{ \int  |A_D(p)|^2 dp}
\end{equation} 

In our analysis, we have the following categories:
\begin{itemize}
\item $\Bm \to (K_S^0 \pi^+ \pi^-)_D \pi^0 \pi^-$
\item $\Bm \to (K_S^0 \pi^+ \pi^-)_D \pi^0 K^-$
\item $\Bm \to (K_S^0 \pi^+ \pi^-)_D \gamma \pi^-$
\item $\Bm \to (K_S^0 \pi^+ \pi^-)_D \gamma K^-$
\item $\Bm \to (K_S^0 \pi^+ \pi^-)_D (\pi^0 \to \gamma) \pi^-$
\item $\Bm \to (K_S^0 \pi^+ \pi^-)_D (\pi^0 \to \gamma) K^-$
\item $\Bp \to (K_S^0 \pi^+ \pi^-)_D \pi^0 \pi^+$
\item $\Bp \to (K_S^0 \pi^+ \pi^-)_D \pi^0 K^+$
\item $\Bp \to (K_S^0 \pi^+ \pi^-)_D \gamma \pi^+$
\item $\Bp \to (K_S^0 \pi^+ \pi^-)_D \gamma K^+$
\item $\Bp \to (K_S^0 \pi^+ \pi^-)_D (\pi^0 \to \gamma) \pi^+$
\item $\Bp \to (K_S^0 \pi^+ \pi^-)_D (\pi^0 \to \gamma) K^+$
\end{itemize}
and the corresponding $K_S^0 K^+ K^-$ modes.
Those with $\pi^0 \to \gamma$ represent the  modes where only one $\gamma$ of the $\pi0$ decay is reconstructed and used for $B^-$ reconstruction.

The observed number of events in each Dalitz bin $i$ and in each modes are expressed as
\begin{eqnarray}
N_i(\Bm, \pi^0, K)  &=&  h'(\Bm, \pi^0, K) [ F'_i  + r^2_B(K)F'_{\bar{i}}   + 2\sqrt{F'_i F'_{\bar{i}} } (x_-(K)c_i  + y_-(K)s_i ) ]  \\\nonumber
N_i(\Bm, \gamma, K)  &=& h'(\Bm, \pi^0, K)[ F'_i  + r^2_B(K)F'_{\bar{i}}  - 2\sqrt{F'_i F'_{\bar{i}} } (x_-(K)c_i + y_-(K)s_i  ) ] \\\nonumber
N_i(\Bm, \pi^0 \to \gamma, K)  &=& h'(\Bm, \pi^0 \to \gamma, K)[ F'_i  + r^2_B(K)F'_{\bar{i}}  + 2\sqrt{F'_i F'_{\bar{i}} } (x_-(K)c_i + y_-(K)s_i  ) ] \\\nonumber
N_i(\Bm, \pi^0, \pi)  &=&  h'(\Bm, \pi^0, \pi)[ F'_i  + r^2_B(\pi)F'_{\bar{i}}   + 2\sqrt{F'_i F'_{\bar{i}} } (x_-(\pi)c_i  + y_-(\pi)s_i ) ] \\\nonumber
N_i(\Bm, \gamma, \pi)  &=& h'(\Bm, \pi^0, \pi)[ F'_i  + r^2_B(\pi)F'_{\bar{i}}  - 2\sqrt{F'_i F'_{\bar{i}} } (x_-(\pi)c_i + y_-(\pi)s_i  ) ] \\\nonumber
N_i(\Bm, \pi^0 \to \gamma, \pi)  &=& h'(\Bm, \pi^0 \to \gamma, \pi)[ F'_i  + r^2_B(\pi)F'_{\bar{i}}  + 2\sqrt{F'_i F'_{\bar{i}} } (x_-(\pi)c_i + y_-(\pi)s_i  ) ] \\\nonumber 
N_i(\Bp, \pi^0, K)  &=&  h'(\Bp, \pi^0, K)[ F'_{\bar{i}}  + r^2_B(K)F'_{i}   + 2\sqrt{F'_i F'_{\bar{i}} } (x_+(K)c_i  - y_+(K)s_i ) ]  \\\nonumber
N_i(\Bp, \gamma, K)  &=& h'(\Bp, \pi^0, K)[ F'_{\bar{i}}  + r^2_B(K)F'_i  - 2\sqrt{F'_i F'_{\bar{i}} } (x_+(K)c_i - y_+(K)s_i  ) ] \\\nonumber
N_i(\Bp, \pi^0 \to \gamma, K)  &=& h'(\Bp, \pi^0 \to \gamma, K)[ F'_{\bar{i}}  + r^2_B(K)F'_i  + 2\sqrt{F'_i F'_{\bar{i}} } (x_+(K)c_i - y_+(K)s_i  ) ] \\\nonumber
N_i(\Bp, \pi^0, \pi)  &=&  h'(\Bp, \pi^0, \pi)[ F'_{\bar{i}}  + r^2_B(\pi)F'_i   + 2\sqrt{F'_i F'_{\bar{i}} } (x_+(\pi)c_i  - y_+(\pi)s_i ) ]  \\\nonumber
N_i(\Bp, \gamma, \pi)  &=& h'(\Bp, \pi^0, \pi)[ F'_{\bar{i}}  + r^2_B(\pi)F'_i  - 2\sqrt{F'_i F'_{\bar{i}} } (x_+(\pi)c_i - y_+(\pi)s_i  ) ] \\\nonumber
N_i(\Bp, \pi^0 \to \gamma, \pi)  &=& h'(\Bp, \pi^0 \to \gamma, \pi)[ F'_{\bar{i}}  + r^2_B(\pi)F'_i  + 2\sqrt{F'_i F'_{\bar{i}} } (x_+(\pi)c_i - y_+(\pi)s_i  ) ] \\\nonumber 
 \end{eqnarray} 
 We make the following assumptions to simply the analysis when considering the dependence of $F'_i$, $c'_i$ and $s'_i$ with efficiency
\begin{itemize}
\item $c'_i$ and $s'_i$ are more robust under efficiency variance and we consider them to be the same for all the modes (will be checked using MC).
\item We use directly $c_i$ and $s_i$ from CLEO-c and the effects due to different efficiency map will be considered in systematics.
\item $F'_i$ are assumed to be the same for $\Bm$ and $\Bp$ modes (trivial assumption). 
\item $F'_i$ are assumed to be the same for $K$ and $\pi$ modes (will be checked using MC). 
\item $F'_i$ are assumed to be the same for $\pi^0$, $\gamma$ and $\pi^0 \to \gamma$ modes (a little non-trivial, need to be checked by MC, if not, need to think about reasons).
\end{itemize}

The normalization factors are related by
\begin{eqnarray}
\frac{h'(B, \pi^0, h)}{h'(B, \gamma, h)} &=&\frac{\BR(D^{*0} \to D^0 \pi^0) \epsilon_{\textrm{ave}}(B, \pi^0, h)}{\BR(D^{*0} \to D^0 \gamma) \epsilon_{\textrm{ave}}(B, \gamma, h)} \\
\frac{h'(B, \pi^0, h)}{h'(B, \pi^0 \to \gamma, h)} &=&\frac{ \epsilon_{\textrm{ave}}(B, \pi^0, h)}{ \epsilon_{\textrm{ave}}(B, \pi^0 \to \gamma, h)} \\
\frac{h'(B, h^0, K)}{h'(B, h^0, \pi)} &=&\frac{\BR(B^- \to D^{*0} K^-) \epsilon_{\textrm{ave}}(B, h^0, h)}{\BR(B^- \to D^{*0} \pi^-) \epsilon_{\textrm{ave}}(B, h^0, \pi)} \\
\end{eqnarray}
We can further replace $h'$ by the total number of events observed in each mode
\begin{eqnarray}
h'(\Bm, \pi^0, K) &=& \frac{N(\Bm, \pi^0, K)}{1+r_B^2(K) + 2(c x_-(K) +  s y_-(K))}\\\nonumber 
h'(\Bm, \pi^0 \to \gamma, K) &=& \frac{N(\Bm, \pi^0 \to \gamma, K)}{1+r_B^2(K) + 2(c x_-(K) +  s y_-(K))}\\\nonumber 
h'(\Bm, \gamma, K) &=& \frac{N(\Bm, \gamma, K)}{1+r_B^2(K) - 2(c x_-(K) +  s y_-(K))}\\\nonumber 
h'(\Bm, \pi^0, \pi) &=& \frac{N(\Bm, \pi^0, \pi)}{1+r_B^2(\pi) + 2(c x_-(\pi) +  s y_-(\pi))}\\\nonumber 
h'(\Bm, \pi^0 \to \gamma, \pi) &=& \frac{N(\Bm, \pi^0 \to \gamma, \pi)}{1+r_B^2(\pi) + 2(c x_-(\pi) +  s y_-(\pi))}\\\nonumber 
h'(\Bm, \gamma, \pi) &=& \frac{N(\Bm, \gamma, \pi)}{1+r_B^2(\pi) - 2(c x_-(\pi) +  s y_-(\pi))}\\\nonumber 
h'(\Bp, \pi^0, K) &=& \frac{N(\Bp, \pi^0, K)}{1+r_B^2(K) + 2(c x_+(K) -  s y_+(K))}\\\nonumber 
h'(\Bp, \pi^0 \to \gamma, K) &=& \frac{N(\Bp, \pi^0 \to \gamma, K)}{1+r_B^2(K) + 2(c x_+(K) -  s y_+(K))}\\\nonumber 
h'(\Bp, \gamma, K) &=& \frac{N(\Bp, \gamma, K)}{1+r_B^2(K) - 2(c x_+(K) -  s y_+(K))}\\\nonumber 
h'(\Bp, \pi^0, \pi) &=& \frac{N(\Bp, \pi^0, \pi)}{1+r_B^2(\pi) + 2(c x_+(\pi) -  s y_+(\pi))}\\\nonumber 
h'(\Bp, \pi^0 \to \gamma, \pi) &=& \frac{N(\Bp, \pi^0 \to \gamma, \pi)}{1+r_B^2(\pi) + 2(c x_+(\pi) -  s y_+(\pi))}\\\nonumber 
h'(\Bp, \gamma, \pi) &=& \frac{N(\Bp, \gamma, \pi)}{1+r_B^2(\pi) - 2(c x_+(\pi) -  s y_+(\pi))}\\\nonumber 
\end{eqnarray}
\begin{eqnarray}
c &=& \sum_i \sqrt{ F'_i F'_{\bar{i}}} c_i  \\
s &=& \sum_i \sqrt{ F'_i  F'_{\bar{i}}} s_i \\
\end{eqnarray}

The above relations for $h'$ are not yet set inside code and can be used to bring more information to the fit.
One could further consider the relationship of $h'$ between $K_S^0 \pi^+ \pi^-$ and $K_S^0 K^+ K^-$ modes (it may not help in add sensitivity, not sure if it will help in stabilizing the fit )
\begin{equation}
\frac{h'(K_S^0 \pi^+ \pi^-)}{h'(K_S^0 K^+ K^-)} = \frac{\BR(D^0 \to K_S^0 \pi^+\pi^-)\epsilon_{\textrm{ave}}(D^0 \to K_S^0 \pi^+ \pi^-)}{\BR(D^0 \to K_S^0 K^+K^-)\epsilon_{\textrm{ave}}(D^0 \to K_S^0 K^+ K^-)}
\end{equation}


The above relationship is valid for the decay $\Bm \to \Dstar \pi^-$ and $\Bm \to \Dstar K^-$. 
The procedure we try to take is to fit $\Bm \to \Dstar \pi^-$ and $\Bm \to \Dstar K^-$ modes together according to the above relationship without any assumptions on the CP violation effects on $\Bm \to \Dstar \pi^-$.
In fact, for each Dalitz bin, we have four measurements, $N(\Bm, h^0, \pi^-)$, $N(\Bp, h^0, \pi^+)$, $N(\Bm, h^0, K^-)$ and $N(\Bp, h^0, K^+)$, that is $4n$ where $n$ is the number of bins in the upper region of Dalitz plot.
while the four equations give us $2n + 12$ unknowns, $F'_i$, $F'_{\bar{i}}$, $x_{\pm}(\pi)$, $x_{\pm}(K)$, $y_{\pm}(\pi)$, $y_{\pm}(K)$, $h'(\Bm, h^0, \pi^-)$, $h'(\Bm, h^0, K^-)$, $h'(\Bp, h^0, \pi^+)$ and $h'(\Bp, h^0, K^+)$.
Even if we don't take into account the relationship between $h'$, when $n > 6$, we have more observables than unknown and thus $\gamma$ can be constrained.
 The lower region of Dalitz plot will add more statistical power ($4n$ new observables) without introducing any unknowns.
